\documentclass[12pt]{article}

\usepackage{graphicx}
\usepackage{longtable}
\usepackage{hyperref}
\usepackage{vhistory}

\renewcommand{\thesection}{}
\renewcommand{\thesubsection}{\arabic{subsection}}

\begin{document}

\begin{titlepage}
	\centering
	{\scshape\LARGE McMaster University \par}
	\vspace{1.5cm}
	{\huge\bfseries Requirements Document \par}
    {\scshape\Large Revision 1 \par}

	\vspace{1cm}
	{\scshape\Large Capstone Team 14\par}
	{\Large\itshape Ananthan Kanagasabai, Andrei Ciontea, Curran Tam, Joseph Nguyen, Victor Siu \par}
	\vspace{3cm}
	\vfill
	supervised by\par
	Dr.Sarah Khan, Wenbo He

	\vfill
	{\large \today\par}
\end{titlepage}

\newpage

\tableofcontents
\listoffigures
\listoftables

\newpage
\section*{Revision History}
\begin{tabular}{|c|c|}
\hline
\textbf{Date}  & \textbf{Comments} \\ \hline
October 12, 2016 & Revision 0 of Requirements Document created \\ 
\hline
April 9, 2017 & Correction and Final Revision\\
\hline
\end{tabular}

\newpage


\section{Project Drivers}
This document was written using the Volere template.

\subsection{The Purpose of the Project }
\subsubsection{The User Business or Background of the Project Effort}
17 HIV Antiretroviral (ARV) medications have been approved for use in children. This results in various regimens available for use, as usually most children need to be on 3 medications at once. There are many factors to consider when initiating ARVs in children as many medications have different toxicities and side effects (including affecting growth, hormones, kidneys etc). Also, certain viruses may be resistant to some medications and not others. Some medications will interact with other HIV or non HIV medications. Some medications come in liquids, dissolvable tablets, or pills which can affect what age children can take them or not. Also, FDA approval for certain medications depends on: age, weight etc. Drug insurance will only cover some medications. Therefore, it is very challenging to decide on a regimen given the multiple permutations.
\subsubsection{Goals of the Project}
By creating a solution for this regimen issue, we can: improve efficiency, reduce human error, and provide an optimal and detailed regimen when medical teams need to decide on prescriptions for HIV patients. Creating this website will also help lessen the time that a doctor needs to use in order to come up with a medication regimen for a child. The doctor will be able to use the saved time for other important tasks and the patient will also receive their prescription at a sooner time.

\subsection{The Client, the Customer, and Other Stakeholders}
\subsubsection{The Client}
The client for this project is Dr. Sarah Khan. She is an assistant professor currently teaching at McMaster University. She is the individual who proposed the project to the team will be the one to test the product and give advice and feedback for possible improvements.
\subsubsection{The Customer}
The software application is being designed for doctors who are treating patients with HIV. The software will be used by users with all skill levels and deliver crucial information to them. Therefore, the application should be simple and fast.
\subsubsection{Other Stakeholders}
The other stakeholders involved with the development of this project are: Dr. Sarah Khan and Dr. Wenbo He. Dr. Khan will be our external supervisor and will be aiding the team with the design of the UI along with the crucial information that is to be implemented in the software. Dr. He will be our internal supervisor and will be aiding the team with the semantics and syntax of the code.

\subsection{Users of the Product}

\subsubsection{The Hands-On Users of the Product}
The only users are doctors taking care of HIV patients. The roles of the doctors are:
\begin{enumerate}
\item Giving patients their medication on time
\item Giving the right medications to patients
\item Giving the correct dosage of the medication
\item Not mixing up the medication and giving the patient alternate medications
\item Completing the above tasks with the algorithm that we will provide
\end{enumerate}

\subsubsection{Priorities Assigned to Users}
{\bf Key Users:} Medical Professionals \newline
{\bf Secondary Users:} Programmers

\subsubsection{User Participation}
\begin{itemize}
\item Users will enter information about their patient into the forms provided
\item Users will look at the provided output of possible medical regimens and select one for their patient
\end{itemize}
\subsubsection{Maintenance Users and Service Technicians}
\begin{itemize}
\item Programmers
\end{itemize}

\section{Project Constraints}
\setcounter{subsection}{3}
\subsection{Mandated Constraints }
\subsubsection{Solution Constraints}
{\bf Description:} The application will provide a page where information about the patient can be inputted in a simple and intuitive way.\\
{\bf Rationale:} Inputting patient information should be easy and fairly quick.\\
{\bf Fit Criterion:} The website will use basic HTML forms that most people in the intended demographic will understand how to use.\\ \\
{\bf Description:} The application will operate on Google Chrome, Internet Explorer, Mozilla Firefox and most likely all other browsers.\\
{\bf Rationale:} After speaking to the medical professional/supervisor, it was understood that a web application would be the most convenient way of using this tool.\\
{\bf Fit Criterion:} The website will be designed using standards set by android studio.

\subsubsection{Implementation Environment of the Current System}
\begin{itemize}
\item The source code will be written in {\bf HTML, JavaScript, PHP, and SQL}
\item {\bf Google Chrome, Mozilla Firefox, and Internet Explorer} will be used to develop and test the application
\item {\bf MySQL} database will be used to store information about each of the HIV medications.
\item  The database and the website will be hosted on Amazon Web Services and will be accessible on all computers
\end{itemize}
\subsubsection{Partner of Collaborative Applications}
N/A
\subsubsection{Off-the-Shelf Software}
N/A
\subsubsection{Anticipated Workplace Environment}
\begin{itemize}
\item Hospitals
\item Doctor's Office
\end{itemize}
\subsubsection{Schedule Constraints}
\begin{itemize}
\item Requirements Document Revision 0: October 26th
\item Requirements Document Revision 1: April 5th
\item Final Demonstration: April 12 2017
\end{itemize}
\subsubsection{Budget Constraints}
N/A

\subsection{Naming Conventions and Definitions}
\subsubsection{Definition of all Terms}
\begin{itemize}
\item {\bf HTML:} Markup language that will be used to create the web pages related to the HIV Regimen Generator.
\item {\bf CSS:} Style sheets used to specify how the HTML pages should look.
\end{itemize}
\subsubsection{Data Dictionary of Any Include Models}
N/A

\subsection{Relevant Facts and Assumptions}
\subsubsection{Facts}
\begin{itemize}
\item Web applications and pages are typically written in HTML.
\item The user interface will be navigated with the use of a keyboard and mouse.
\end{itemize}
\subsubsection{Assumptions}
\begin{itemize}
\item Assume that the user can use a mobile phone or computer to access the Internet
\item The user will be a doctor or a member of the medical field
\item The doctor will be given the choice of selecting a medication timetable from the ones given in our web application
\end{itemize}

\section{Functional Requirements}
\setcounter{subsection}{6}
\subsection{The Scope of the Work}
\subsubsection{The Current Situation}
There are many factors medical teams need to consider when prescribing ARVs to children. Every type of medication has different side effects and toxicities. Some medications may not be efficient to certain viruses because those viruses are resistant. The state of medications (gas,liquid, or solid) are one of the factors that medical teams need to consider because they can determine which age group the medication can be administered to. The medical teams need to compose different groups of medications depending on each patient’s conditions. Therefore, it is very challenging to decide on a regimen given that there are such a vast amount of combinations. Medical professionals have several other tasks that they need to complete in any given day so they cannot be on service all the time. In some emergency cases, the patient needs to receive their prescription at a sooner time. 
\\We intend to build an application that can be functional as a website. The medical professional can take their patient’s medical information and generate the required medications to lessen the time that it takes to manually come up with a medication regimen for a child.

\subsubsection{The Context of the Work}
\newpage
\vfill
\begin{figure}[ht]
  \includegraphics[width=\linewidth]{image00.png}
  \caption{Work Context Diagram}
  \label{fig:Work Context Diagram}
\end{figure}
\vfill

\subsubsection{Work Partitioning}
\begin{longtable}{|p{5cm}|p{5cm}|p{5cm}|}
\hline
Event Name & Input/Output & Summary \\
\hline
Hospital enter patient’s information  & Patient’s information(In) & Store the patient’s information in database \\
\hline
Hospital enter the type and amount of medications & Medication information(In) & Store the medication information in database \\
\hline
User enter user password & User Password(In) & Verify user’s password and user log in \\
\hline
User enter the request & Requests(In) & Accept the user’s request \\
\hline
Send information to user & Patient’s information(out) & Display the patient's information to user \\
\hline
Send information to user & Medication information (out) & Display the medication information to user \\
\hline
User enter medications & Chosen medications(In) & Store the chosen medications in database \\
\hline
Send information to hospital  & Chosen medications(out) & Display the chosen medications to hospital \\

\caption{Work Partitioning}
\label{tab:workpartition}
\end{longtable}

\subsection{The Scope of the Product}
\subsubsection{Product Boundary}
\begin{itemize}
\item The application will be functional only on Android devices.
\end{itemize}

\subsubsection{Product Use Case}
 
\begin{figure}[ht]
  \includegraphics[width=\linewidth]{image01.png}
  \caption{User Case Diagram}
  \label{fig:User Case Diagram}
\end{figure}

\subsection{Functional and Data Requirements}
\subsubsection{Functional Requirements}
{\bf Requirement \# 1} \\
{\bf Description:} The product shall store the patient’s description given by the hospital into the database.  \\
{\bf Rationale:} The input data is required for medical teams. The patient’s condition needs to be considered when prescribing medication. \\
{\bf Fit Criterion:} The database can save the patient’s data from the device in the hospital. \\ \\
{\bf Requirement \# 2}\\
{\bf Description:} The product shall store the medication information given by hospital into database.  \\
{\bf Rationale:} The input data is required for medical teams. Since they need to consider the available medication in the hospital. \\
{\bf Fit Criterion:} The database can save the medication data from the device in hospital. \\ \\
{\bf Requirement \# 3}\\
{\bf Description:} The product shall be able to add, delete and modify the data in database.  \\
{\bf Rationale:} If the user data in the database needs to be updated to the most current version.\\
{\bf Fit Criterion:} The hospital and user can be add, delete and modify the data in database.\\ \\
{\bf Requirement \# 4}\\
{\bf Description:} The product shall be able to display the required information to a user if the user sent the request.\\
{\bf Rationale:}The medical teams need to consider that the medications depend on the patient's condition and the availability of the medication in the hospital.\\
{\bf Fit Criterion:} The device can display the data from the database anytime after sending the request. \\ \\
{\bf Requirement \# 5}\\
{\bf Description:} The product shall be able to display the chosen medications selected by the user.\\
{\bf Rationale:} The hospital needs the medical teams’ order to give the patient the correct medications.\\
{\bf Fit Criterion:}  The device in the hospital can display the chosen medications from the database \\ \\
{\bf Requirement \# 6}\\
{\bf Description:} The product shall be able to allow the user to enter a password.  \\
{\bf Rationale:} To prevent any unauthorized and viewing of personal information of patient. \\
{\bf Fit Criterion:} The device of will allow a user to enter a password.\\ \\
{\bf Requirement \# 7}\\
{\bf Description:} The product shall be able to verify the user’s password. Restrict a user’s access if the password is incorrect.\\
{\bf Rationale:} To prevent any unauthorized and viewing of personal information of patient.\\
{\bf Fit Criterion:} The database cannot allow the person any access without the correct password.\\ \\
{\bf Requirement \# 8}\\
{\bf Description:} The product shall store the chosen medication given by a user into database.  \\
{\bf Rationale:} The doctor can select the medication anytime.\\
{\bf Fit Criterion:} The database can save the input with the chosen medication from the device of the user.


\subsubsection{Data Requirements}
\begin{itemize}
\item Only the users can be allow to access and view the data(patient and medication) in database
\item Only the users and hospital can be allow to view the chosen medication
\item Data in database should always be able to save and load securely and without any error 
\end{itemize}

\section{Nonfunctional Requirements}
\setcounter{subsection}{9}
\subsection{Look and Feel Requirements}
\subsubsection{Appearance Requirements}
\begin{itemize}
\item A group of English speakers in the age range of 25-70 should be able to understand the UI within 10 minutes of using the software. The reason for selecting this particular audience is because it is the approximate age of a doctor in Canada (doctors being the key demographic that the website will be directed towards).
\item The UI should fit in all resolutions of the users computer screen.
\end{itemize}

\subsubsection{Style Requirements}
\begin{itemize}
\item After the first use of the website, most users should feel comfortable navigating through the web pages and should have a positive experience with the style and look of it. This will be tested for.
\end{itemize}

\subsection{Usability and Humanity Requirements}
\subsubsection{Ease of Use Requirement}
\begin{itemize}
\item After a year of using the product, the error rate shall be close to 0 percent
\item The software account registration shall be detailed enough so that a user will always be able to obtain their password in case they forgot it
\end{itemize}

\subsubsection{Personalization and Internalization Requirements}
\begin{itemize}
\item The first design of the UI shall only be available in English.
\item The first design of the software shall only be compatible on the Android OS.
\item The first design of the software shall not allow users to make changes to the UI.
\end{itemize}

\subsubsection{Learning Requirements}
\begin{itemize}
\item Anyone in the age range of 25-70 shall be able to use the UI with little to no help.
\end{itemize}

\subsubsection{Understandability and Politeness Requirements}
\begin{itemize}
\item The software shall use words and symbols that are understandable by the any user in the intended age range.
\end{itemize}
\subsubsection{Accessibility Requirements}
N/A

\subsection{Performance Requirements}
\subsubsection{Speed and Latency Requirements}
\begin{itemize}
\item The application speed will vary depending on the Android operating system, but will generally respond within a second. Latency is not relevant as our application does not use any network.
\end{itemize}

\subsubsection{Safety-Critical Requirements}
\begin{itemize}
\item Private information about patients on accounts will only be viewed by the users.
\item Only the users may access and modify the medical information of a user.
\end{itemize}

\subsubsection{Precision or Accuracy Requirements}
\begin{itemize}
\item The app will manage medication schedule for patients down to the time of day. 
\item It will take into account the patient's physical condition as well as compatibility with other medications when constructing this schedule.
\end{itemize}

\subsubsection{Reliability and Availability Requirements}
\begin{itemize}
\item The app is tied to the user’s mobile device and should be available at any time.
\end{itemize}

\subsubsection{Robustness or Fault-Tolerance Requirements}
\begin{itemize}
\item A patient’s medication schedule should respond and/or change appropriately when a new medication is added based on its compatibility with existing medications.
\end{itemize}

\subsubsection{Capacity Requirements}
\begin{itemize}
\item As this app is independent to each user, there is no capacity to the number of users.
\end{itemize}

\subsubsection{Scalability or Extensibility Requirements}
\begin{itemize}
\item Users will be able to add additional medication to their account, and the schedule be changed accordingly.
\end{itemize}

\subsubsection{Longevity Requirements}
\begin{itemize}
\item The application will be able to function indefinitely until there are new major medical discoveries that drastically change the way current medication should be taken.
\end{itemize}

\subsection{Operational and Environmental Requirements}

\subsubsection{Expected Physical Environment}
\begin{itemize}
\item As long as the user is in possession of their mobile device, the app can be used anywhere.
\end{itemize}

\subsubsection{Requirements for Interfacing with Adjacent Systems}
N/A

\subsubsection{Productization Requirements}
\begin{itemize}
\item The app will be available for download from the Google Play store.
\end{itemize}

\subsubsection{Release Requirements}
\begin{itemize}
\item New releases of the app will be based on new medication’s compatibility with the old.
\end{itemize}

\subsection{Maintainability and Support Requirements}

\subsubsection{Maintenance Requirements}
N/A

\subsubsection{Supportability Requirements}
\begin{itemize}
\item The software shall be hard-coded and not supported. The code will still be accessible for the team in case any changes need to be made. 
\end{itemize}

\subsubsection{Adaptability Requirements}
\begin{itemize}
\item The software shall operate on the Android OS.
\end{itemize}

\subsection{Security Requirements}

\subsubsection{Access Requirements}
\begin{itemize}
\item A username and password will not be required to access the data on the website. The info stored on the website is not user-sensitive and will not reveal information about the doctors or the patients that are involved. The database will store information about medications that are available to the public and the generator will not save information inputted by the users.
\end{itemize}

\subsubsection{Integrity Requirements}
\begin{itemize}
\item The software shall be guarded from any misuse.\\
\item The software shall be protected from any local or remote attack.
\end{itemize}

\subsubsection{Privacy Requirements}
\begin{itemize}
\item The software shall not send any data remotely.
\end{itemize}

\subsubsection{Audit Requirements}
N/A

\subsubsection{Immunity Requirements}
N/A

\subsection{Cultural and Political Requirements}
\subsubsection{Cultural Requirements}
N/A
\subsubsection{Political Requirements}
N/A

\subsection{Legal Requirements}
\subsubsection{Compliance Requirements}
N/A
\subsubsection{Standards Requirements}
N/A

\section{Project Issues}
\setcounter{subsection}{17}
\subsection{Open Issues}
\begin{itemize}
\item The government constantly updates the list of approved medication for use by children. Our implementation will include the most updated list at the time of release. The database will be easily accessible and can be easily modified. With the proper tools, new medications can be added and old medications can be removed. 
\end{itemize}

\subsection{Off-the-Shelf Solutions }
Currently, our medication regimen application seems to be the only off-the-shelf solution for our client. There does not exist any solution outside of manual sorting/organizing to help filter out medication constraints for the HIV patients and provide them with the ideal dosage and schedule.

\subsection{New Problems }
We aim to have our application completely stand-alone while referencing medical information provided publicly by the government. The application should not interfere with any operations in hospitals.

\subsection{Tasks}
\begin{itemize}
\item Revise requirements document.
\item Create a test plan.
\item Demonstrate a proof of concept.
\item Draw up design documents.
\item Revision 0 project demonstration.
\item Create a user guide for the project.
\item Write up a test report.
\item Final revision project demonstration.
\item Write final revisions to documentation.
\end{itemize}

\subsection{Migration to the New Product }
N/A

\subsection{Risks}
\begin{itemize}
\item If the algorithm is not carefully implemented, it might generate faulty regimens that doctors may overlook and suggest to patients. This could lead to serious consequences.
\item User passwords might be cracked even under reinforced encryption methods. The affected victims may blame the development team for not creating a secure system.
\item There will be no way to access the database with existing application credentials if the user forgets the application password.
\end{itemize}

\subsection{Costs}
There are no direct monetary costs associated with this project, but about half a year of development time will be required.

\subsection{User Documentation and Training}
Users will be provided with information on the program use via a “Help” option which, when selected, will open a dialog box detailing general functionality of the program along with an FAQ section. Beyond the help document, a user’s familiarity with casual computer use should require no further training.

\subsection{Waiting Room}
\begin{itemize}
\item Currently, there are no plans on making a web-based version of the android application. However, this is something to consider upon future releases of our product.
\item Medical information is planned to be implemented statically. Dynamic implementation can be considered for keeping the information up-to-date  and automated by our clients.
\end{itemize}

\subsection{Ideas for Solutions}
\begin{itemize}
\item Main programming language should be Java and coded inside Android Studio
\end{itemize}




\end{document}